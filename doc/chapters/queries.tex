\chapter{Queries}
In this chapter, we present only the complex queries executed by our application, without dwelling on trivial queries. 

\section{Mongo DB}
\subsection{Search Query}
This query is run every time a guest user, a patient or a doctor inserts a character in the search bar. t works in the same way for all three actors. Returns a list of doctors based on the entered string. The string can be a name, a specialization, a city, a province or a mix of these. 

This is the Java code:
\begin{lstlisting}[language=java]
	@Override
	public List<DoctorMongoProjection> searchDoctors(String text) {
		// Match documents using full-text search
		AggregationOperation match = match(Criteria.where("$text").is(new Document("$search", text)));
		
		// Add MongoDB's built-in text relevance score
		AggregationOperation project = context -> new Document("$project",
		new Document("doctor", "$$ROOT")
		.append("score", new Document("$meta", "textScore"))
		);
		
		// Sort by relevance score
		AggregationOperation sort = context -> new Document("$sort",
		new Document("score", new Document("$meta", "textScore"))
		);
		
		// Limit to top 250 results
		AggregationOperation limit = limit(250);
		
		// Build the aggregation pipeline
		Aggregation agg = newAggregation(match, project, sort, limit);
		
		// Execute aggregation
		AggregationResults<DoctorMongoProjection> results =
		mongoTemplate.aggregate(agg, "doctors", DoctorMongoProjection.class);
		
		return results.getMappedResults();
	}
\end{lstlisting}

This is its one-to-one translation into a MongoDB query. 
\begin{lstlisting}[language=mongodb]
	db.doctors.aggregate([
		{
			$match: {
				$text: { $search: query }
			}
		},
		{
			$project: {
				doctor: "$$ROOT",
				score: { $meta: "textScore" }
			}
		},
		{
			$sort: {
				score: { $meta: "textScore" }
			}
		},
		{
			$limit: 250
		}
	]);
\end{lstlisting}

The above aggregation pipeline enables a full-text search over the \texttt{doctors} collection, returning the most relevant documents according to a weighted scoring scheme. Although the underlying text index will be discussed in detail in the dedicated indexing section, it is important to mention that the fields \texttt{name}, \texttt{specializations}, \texttt{address.city}, and \texttt{address.province} are indexed with different weights, thereby influencing the ranking of the search results.

The pipeline is composed of four main stages:
\begin{itemize}
	\item \textbf{\texttt{\$match}}: This stage filters documents that match the full-text search criteria specified by the user input. The search is performed across all indexed fields using MongoDB's \texttt{\$text} operator;
	
	\item \textbf{\texttt{\$project}}: At this stage, each document is projected into a new structure where the original document is nested under the \texttt{doctor} field. Additionally, a new field \texttt{score} is introduced, which represents the text relevance score computed by MongoDB using its internal ranking algorithm based on term frequency and field weights;
	
	\item \textbf{\texttt{\$sort}}: Documents are then ordered in descending order of their \texttt{score}, so that the most relevant matches (according to the text index configuration) are prioritized in the output;
	
	\item \textbf{\texttt{\$limit}}: Finally, the result set is truncated to a maximum of 250 documents, which serves both as a performance optimization and a practical upper bound for rendering results in the user interface.
\end{itemize}

\subsection{Analytics}
\subsubsection{Earnings}
This query is an analytic for the doctor's Dashboard and returns a map of monthly revenues for a given year.

\begin{lstlisting}[language=java]
	@Override
	public Map<String, Double> getEarningsByYearForDoctor(String doctorId, Integer year) {
		
		MatchOperation matchOperation = Aggregation.match(Criteria.where("doctor.id").is(doctorId)
		.and("date")
		.gte(LocalDate.of(year, 1, 1).atStartOfDay())
		.lt(LocalDate.of(year + 1, 1, 1).atStartOfDay())
		);
		ProjectionOperation projectMonth = Aggregation.project().andExpression("month(date)").as("month").and("price").as("price");
		GroupOperation groupOperation = Aggregation.group("month").sum("price").as("total");
		ProjectionOperation projectionOperation = Aggregation.project("total").and("month").previousOperation();
		
		Aggregation aggregation = Aggregation.newAggregation(
		matchOperation,
		projectMonth,
		groupOperation,
		projectionOperation
		);
		
		AggregationResults<DBObject> results = mongoTemplate.aggregate(aggregation, Appointment.class, DBObject.class);
		
		Map<String, Double> earningsByYear = new HashMap<>();
		for (DBObject doc : results.getMappedResults()) {
			Integer month = (Integer) doc.get("month");
			String monthString = new DateFormatSymbols(Locale.ENGLISH).getMonths()[month-1].toLowerCase();
			Double total = (doc.get("total") instanceof Number) ? ((Number) doc.get("total")).doubleValue() : null;
			earningsByYear.put(monthString, total);
		}
		
		return earningsByYear;
	}
\end{lstlisting}

\begin{lstlisting}[language=mongodb]
	db.appointments.aggregate([
	{
		$match: {
			"doctor._id": ObjectId(doctorId),
			date: {
				$gte: ISODate("${year}-01-01T00:00:00Z"),
				$lt: ISODate("${year + 1}-01-01T00:00:00Z")
			}
		}
	},
	{
		$project: {
			month: { $month: "$date" },
			price: 1
		}
	},
	{
		$group: {
			_id: "$month",
			total: { $sum: "$price" }
		}
	},
	{
		$project: {
			month: "$_id",
			total: 1,
			_id: 0
		}
	}
	]); 
\end{lstlisting}

The aggregation pipeline shown above is designed to compute the monthly revenue generated by a specific doctor over the course of a given calendar year. This query supports the analytical needs of the doctor's dashboard, enabling a temporal breakdown of financial performance.

The pipeline consists of the following sequential stages:

\begin{itemize}
	\item \textbf{\texttt{\$match}}: This stage performs a selective filtering of documents within the \texttt{appointments} collection. It restricts the result set to only those documents where the embedded \texttt{doctor.\_id} field matches the provided identifier and the \texttt{date} field falls within the specified year. The temporal filtering is implemented using the \texttt{\$gte} and \texttt{\$lt} operators to define an inclusive-exclusive date range between the first day of January and the first day of the following year;
	
	\item \textbf{\texttt{\$project}}: Each document is then transformed to retain only two fields: the \texttt{price} of the appointment and a computed \texttt{month} field, which is extracted from the \texttt{date} using MongoDB’s \texttt{\$month} operator. This transformation facilitates the grouping operation in the subsequent stage;
	
	\item \textbf{\texttt{\$group}}: Documents are aggregated by the computed \texttt{month} value. For each month, the total revenue is computed by summing the values of the \texttt{price} field using the \texttt{\$sum} accumulator. The grouping key, \texttt{\_id}, is implicitly set to the month number (ranging from 1 to 12);
	
	\item \textbf{\texttt{\$project}}: In the final projection stage, the output is restructured to explicitly include the \texttt{month} and \texttt{total} fields, while omitting the default \texttt{\_id} field. This renders the result set suitable for direct transformation into a month-to-amount mapping.
\end{itemize}

The result of this pipeline is a flat collection of documents, each representing the total earnings for a specific month. In the application layer, this output is subsequently mapped into a \texttt{Map<String, Double>}, where each key corresponds to the English name of a month (e.g., \texttt{"january"}, \texttt{"february"}), and each value denotes the corresponding revenue in that month. This processed result can be directly visualized in the dashboard through line charts.

\subsubsection{Visit Type Summary}
...

\begin{lstlisting}[language=java]
	
\end{lstlisting}

\begin{lstlisting}[language=mongodb]
	
\end{lstlisting}

\section{Indexes}

\subsection{MongoDB}
MongoDB supports a variety of index types, each designed to optimize query performance for specific use cases. The following sections detail the indexes implemented in our MongoDB collections, which are crucial for enhancing the efficiency of our application's data retrieval operations.

\subsubsection{Appointments Collection - idx\_patient\_id}

\begin{table}[H]
\centering
\caption{Performance comparison with and without the \texttt{idx\_patient\_id} index}
\begin{tabular}{lccc}
\toprule
\textbf{Metric} & \textbf{With index} & \textbf{Without index} & \textbf{\% Change} \\
\midrule
Execution time (ms)         & 333.20     & 892.80     & \textbf{-62.68\%} \\
Documents examined (avg)         & 4,634.60   & 699,989.00 & \textbf{-99.34\%} \\
Index keys examined (avg)        & 4,634.60   & 0.00       & N/A \\
Documents returned (avg)         & 4,634.60   & 4,634.60   & +0.00\% \\
\bottomrule
\end{tabular}
\label{tab:patient_idx_performance_corrected}
\end{table}

The \texttt{idx\_patient\_id} index significantly improves query performance when retrieving appointments for patients with a high number of visits. The average execution time decreased by more than 60\%, and the number of documents examined dropped by over 99\%. This demonstrates the importance of indexing frequently queried fields.

\subsubsection{Appointments Collection - idx\_doctor\_id}


\begin{table}[H]
\centering
\caption{Performance comparison with and without the \texttt{idx\_doctor\_id} index}
\begin{tabular}{lccc}
\toprule
\textbf{Metric} & \textbf{With index} & \textbf{Without index} & \textbf{\% Change} \\
\midrule
Execution time (ms)         & 13.70      & 566.80     & \textbf{-97.58\%} \\
Documents examined (avg)         & 1,076.00   & 699,989.00 & \textbf{-99.85\%} \\
Index keys examined (avg)       & 1,076.00   & 0.00       & N/A \\
Documents returned (avg)         & 5.90       & 5.90       & +0.00\% \\
\bottomrule
\end{tabular}
\label{tab:doctor_idx_performance}
\end{table}

Similarly, the \texttt{idx\_doctor\_id} index is essential for queries involving doctor-related appointment data, such as revenue analytics. The index reduced execution time by nearly 98\% and minimized the documents examined from nearly 700,000 to just over 1,000, highlighting a dramatic gain in efficiency.

\subsubsection{Doctors Collection - DoctorsTextIndex}


\begin{table}[H]
\centering
\caption{Performance comparison between regex-based and text index-based search}
\begin{tabular}{lccc}
\toprule
\textbf{Metric} & \textbf{Text Search} & \textbf{Regex Search} & \textbf{\% Change} \\
\midrule
Execution time (ms)         & 108.29     & 274.14     & \textbf{-60.50\%} \\
Documents examined (avg)         & 14,081.00  & 87,632.00  & \textbf{-83.93\%} \\
Index keys examined (avg)        & 31,073.71  & 0.00       & N/A \\
Documents returned (avg)         & 14,081.00  & 9,587.71   & \textbf{+46.87\%} \\
\bottomrule
\end{tabular}
\label{tab:text_vs_regex}
\end{table}

We compared regex-based search with full-text indexed search using \texttt{DoctorsTextIndex}. The indexed search not only executed over 60\% faster but also returned 47\% more documents on average, with significantly fewer documents examined. This underlines the advantage of using text indexes for flexible and performant search capabilities.

\subsection{Neo4j}


During the initial setup of the Neo4j database, the insertion of relationships between \texttt{User} (patients) and \texttt{Doctor} nodes was significantly hindered by the lack of indexing. The script responsible for importing the data failed to complete within a reasonable time. By introducing indexes on the \texttt{id} field of both node types—which duplicates the original MongoDB \_id—relationship creation was drastically accelerated. After indexing, the setup process, which previously stalled indefinitely, was completed in just a few seconds. This confirms the critical role of indexes even in graph databases when handling large-scale relationship insertions.
